\documentclass[a4paper,11pt]{scrartcl}
\usepackage[T1]{fontenc}
\usepackage[latin1]{inputenc}
\usepackage{alltt}
\usepackage{times}
\usepackage[pdftex,
  colorlinks=true,
  linkcolor=blue
]{hyperref}

\title{imc2ctmdp - Tool to transform IMC to CTMDP}
\author{Clemens Hammacher (hamcle@studcs.uni-sb.de)}
\date{\today}

\newcommand{\imcToCtmdp}{\textbf{\textit{imc2ctmdp}} }
\newcommand{\dd}{-- --}

\begin{document}

\maketitle
\tableofcontents
\newpage

\section{Purpose of this document}
This document describes the usage of the \imcToCtmdp Tool, that's mainly used
to transform an IMC (interactive markov chain) into a CTMDP (continuous-time
markov decision process), and has been developed at the Dependable Systems
and Software group\cite{depend-group} of the Saarland University.
It lists and describes the available input formats,
output formats, and command-line options. The transformation in general is
described in a dedicated pdf-file\cite{paper-transformation}.


\section{Input formats}

In the following, we describe the different input formats, that \imcToCtmdp can
handle.

\subsection{the BCG format}\label{input-bcg}

BCG stands for \textit{binary coded graph}. A BCG file contains states and
labelled transitions.
Transitions are interpreted as markov transitions, if they contain the string
``rate ''

There is a library needed in order to read BCGs, called CADP,
see \cite{bcg-page}.

\subsection{the PRISM format}\label{input-prism}

A PRISM file is a model description in a kind of ``programming language'', that
can be interpreted, checked and exported by the PRISM software.

For a description of the language, see \cite{prism-page}.

In order to read such a PRISM file, there must exist two additional files
that contain the state labels and the transitions. These files can be generated
by the PRISM software, using the switches ``-exportlabels'' and
``-exporttrans''.

To introduce interactive transitions, we define special comments in the PRISM
file. For example, the line\\
\texttt{const int act1 = 1001; // Action ``Action1''}\\
tells the parser, that every occuring transition of the rate 1001 should be
replaced by an interactive transition with the label ``Action1''.
You may also use the data type ``double'' instead of ``int'' or add additional space (blanks or tabs).


\section{Output formats}

In the following, we describe the different output formats that \imcToCtmdp can handle.

\subsection{the ETMCC format}\label{output-etmcc}

This is the format used by the ETMCC model checker \cite{etmcc-page}.
There will be two files written: The first one, that usually takes the
extension ``.tra'', contains the transitions, the second one (see \ref{output-lab}) the state labels.

The header of the ``tra''-file contains the number of states and transitions,
in the following form:
\begin{verbatim}
STATES <nrOfStates>
TRANSITIONS <nrOfTransitions>
\end{verbatim}

After the header, there will be a list of all transitions, one per line.
They are ordered by the number of the emanating state. The states are numbered
from 1 on (i.e. the last state has the number <nrOfStates>).
The transitions are listed as follows:
\begin{verbatim}
<action> <sourceStateNr> <targetStateNr> <rate> <type>
\end{verbatim}
The action is ``d'' for interactive transitions and ``r'' for markov transitions.
The rate of interactive transitions is ``0.0''.
The type is ``I'' for interactive (immediate) transitions and ``M'' for markov transitions.

Pay attention to the fact that this file contains markov and interactive
transitions, i.e. in fact it describes a strictly alternating IMC, and no
CTMDP.

\subsection{the CTMDP formats}\label{output-ctmdp}

The CTMDP format is very similar to the ETMCC format. It will also create two
files, one for the transitions (``.ctmdp'' or ``.ctmdpi'') and one for the
state labels (see \ref{output-lab}).

The transitions-file begins with the header, that contains the number of states
and a list of all actions.
\begin{verbatim}
STATES <nrOfStates>
#DECLARATION
<firstLabel>
<secondLabel>
...
#END
\end{verbatim}

After the header, a list of all transitions is printed. It is different for ``.ctmdp'' and ``.ctmdpi'' files, and is described below.

\subsubsection{.ctmdp}

The lines have the form
\begin{verbatim}
<sourceStateNr> <targetStateNr> <action> <propability>
\end{verbatim}

Remember that there may be multiple lines with the same <sourceStateNr> and the same <action>.

\subsubsection{.ctmdpi}

The lines have the form
\begin{verbatim}
<sourceStateNr> <action>
* <targetStateNr1> <propability1>
* <targetStateNr2> <propability2>
...
\end{verbatim}

Remember that this format can describe internal nondeterminism, when there are
multiple entries with the same <sourceStateNr> and <action>.

If there is no internal nondeterminism, this situation will not occure.

\subsection{the .lab file}\label{output-lab}

The ``.lab''-file describes the state labels.
Just like the other formats, it consists of a header and a list of state
descriptions.

The header enumerates all labels used and looks like
\begin{verbatim}
#DECLARATION
<firstLabel>
<secondLabel>
...
#END
\end{verbatim}
where - at the moment - only the two labels ``reach'' and ``absorbing'' may
occure.

After that, there is a list of all labelled states, in the form
\begin{verbatim}
<stateNr> <label> [<label> [<label> ...]]
\end{verbatim}

Unlabelled states do not occure in the list.


\section{Command-line options}

\subsection{specifying files}

When specifying files, the format of the file is extracted from the extension
(i.e. a file named ``test.ctmdp'' is assumed to be in ``ctmdp'' format).
If you want to override this, you may give the format in front of the filename
and separated by a colon (i.e. ``ctmdp:testfile.txt'' for the file
``testfile.txt'' in ``ctmdp'' format).

If no format can be extracted from the filename, it is assumed to be ``bcg'',
and ``.bcg'' is appended as extension.


\subsubsection{\label{sec:input-file}specifying the input file}

The input file can either be given just behind the other command-line
options, or by the parameter ``-i'' (or ``\dd input'').
The following commands are equal:
\begin{verbatim}
./imc2ctmdp test.bcg
./imc2ctmdp -i test.bcg
./imc2ctmdp -itest.bcg
./imc2ctmdp --input=test.bcg
\end{verbatim}

\subsubsection{\label{sec:output-files}specifying the output files}

The output filenames can be given as comma-separated list by the option ``-o''
(or ``\dd output'').
If no output filenames are specified, the output file will be in BCG format
and will have the name of the input file without extension and extended by
``\_ctmdp.bcg''.

If you just give the format, and no filename (e.g. ``ctmdp:''), the filename
will be the input filename without extension, appended by ``\_ctmdp'' and the
format as extension (i.e. ``cmtpd:'' is the same as ``myIMC\_ctmdp.ctmdp'',
if the input filename was ``myIMC.bcg'').

\subsection{\label{sec:input-options}input-related options}

There are two options concerning the input file: ``-k'', or ``\dd no-cycle-search'',
disables searching for interactive cycles while reading the input file.
Only use this option if you are sure that there are no interactive cycles,
because otherwise, the program may hang or crash.
The other option is ``-n'', or ``\dd no-uniformize'', that disables uniformizing the
IMC after reading it from the input file. If the IMC is not uniform, then
most probably the CTMDP will not be uniform too. In this case, a warning is
printed.

\subsection{\label{sec:output-options}output-related options}

The only option relating the output files is ``-s'' (or ``\dd search-absorbing'').
This option solely affects the .lab file (\ref{output-lab}), so using it only
makes sense if there is at least one .lab file written.
The option enables searching for (and labelling) ``absorbing states'', i.e.
a kind of ``deadlock'' states. In IMC, that is an interactive state, whose
outgoing transitions only lead to markov states, whose outgoing transitions
lead back to the interactive state. Longer ``absorbing cycles'' are not found.

\subsection{\label{sec:transformation-options}options concerning the transformation}

The main option for the transformation is ``-a'', or ``\dd action''. This option sets
the transition label that is used to indicate ``interesting'' states. Every
read transition labelled with that label is removed, and the emanating state
is marked. These marked states are preserved through the transformation, and
are marked as ``reach'' in output .lab-files.

Another option is ``-d'', or ``\dd delete-unreachable'', that tells the program to
search for unreachable states and remove them. If there exist such states,
this option may decrease the execution time of the transformation, but of
course the check itself needs some time. The check is done twice: the first time
just before the transformation, and another time just after the transformation,
because on weird IMCs, the transformation may produce unreachable states.

\subsection{alphabetical index}

\begin{itemize}

\item --a, \quad \dd action\\
      Sets the label that markes the interesting states.

\item --c, \quad \dd no-color\\
      Do not use colored output.

\item --d, \quad \dd delete-unreachable\\
      Search for unreachable states and remove them.

\item --i, \quad \dd input=filename[.bcg]\\
      Specifies the input filename (see \ref{sec:input-file}). If no extension is given, ".bcg"
      will be appended.

\item --k, \quad \dd no-cycle-search\\
      Don't search for interactive cycles (only use this option if you know
      that no interactive cycles exist, otherwise the program may fail).

\item --l, \quad \dd no-labels\\
      Don't compute the labels of transitions to the markov successors of interactive states.
      If this is set, all action labels will be ``DFS''.

\item --n, \quad \dd no-uniformize\\
      Per default, the IMC is uniformized. By this option, you can
      disable this feature.

\item --o, \quad \dd output=filenames\\
      Specifies the output files (see \ref{sec:output-files}).

\item --s, \quad \dd search-absorbing\\
      Search for absorbing (``deadlock'') states and label them in the .lab
      file.

\item --h, \quad \dd help\\
      Print a little help.

\end{itemize}

\section{Examples of usage}

Some of the examples use other tools, like \emph{PRISM}\cite{prism-page},
\emph{MRMC}\cite{mrmc-page} or the tool \emph{prism2bcg}, that may have been
provided together with \imcToCtmdp.

\subsection{simple transformation from BCG to BCG}

This is the simplest thing you can do. If the interesting action is ``action'',
you can type one of the following (the first three are equal):
\begin{verbatim}
./imc2ctmdp -a action input.bcg
./imc2ctmdp -i input -a action
./imc2ctmdp -a action -o bcg: input.bcg
./imc2ctmdp -a action -o different_output.bcg \
    input.bcg
\end{verbatim}

\subsection{PRISM file to MRMC format}

A PRISM file in general contains only markov transitions.
In \ref{input-prism}, we introduced a way to define interactive transitions
in a PRISM file. Now we want to check for the action ``critical''.
To make it more interesting, we use a PRISM model (cluster.sm) that needs a
constant (N) to be set.
Now there are two differing ways to transform this file (which holds an IMC)
into a CTMDP in -- let's say -- .ctmdpi format.

The first one makes use of the original PRISM software, the second one uses
the tool prism2bcg, that includes the PRISM library and is a little bit more
handy.

\subsubsection{using PRISM}

\begin{enumerate}

\item Create the .trans and .labels files

These files are needed by \imcToCtmdp to read the PRISM file. They are created
by the PRISM software, using the switches ``-exportlabels'' and
``-exporttrans'', and must have exactly the same filename as the PRISM file,
except for the extension, that must be ``.labels'' and ``.trans'' respectively.
\begin{verbatim}
$PRISM_PATH/bin/prism \
    -exportlabels cluster.labels \
    -exporttrans cluster.trans \
    -const N=32 \
    cluster.sm
\end{verbatim}

\item Transformation

Now you can use the file cluster.sm as input file for \imcToCtmdp. It will also
read the two files that we just created.
\begin{verbatim}
./imc2ctmdp -a critical -o ctmdpi: cluster.sm
\end{verbatim}
The generated output files are cluster\_ctmdp.ctmdpi and cluster\_ctmdp.lab.
These files can directly be read by MRMC.

If you prefer other output filename, you could use for example
\begin{verbatim}
./imc2ctmdp -a critical -o clust32.ctmdpi \
    cluster.sm
\end{verbatim}
which will create clust32.ctmdpi and clust32.lab.

\end{enumerate}

\subsubsection{using prism2bcg}

\begin{enumerate}

\item Converting PRISM to BCG

Using the tool \emph{prism2bcg}, you can easily convert the PRISM file to a
BCG file, that contains all information (e.g. the interactive transitions).
\begin{verbatim}
$PRISM2BCG_DIR/bin/prism2bcg -const N=32 \
    cluster.sm
\end{verbatim}
This creates the file cluster.bcg. You may again want to set another output
file:
\begin{verbatim}
$PRISM2BCG_DIR/bin/prism2bcg -const N=32 \
    -o clust32.bcg cluster.sm
\end{verbatim}

\item Transforming BCG to .ctmdpi

Now you can just use the cluster.bcg as input file for \imcToCtmdp and create
the desired .ctmdpi file.
\begin{verbatim}
./imc2ctmdp -a critical -o ctmdpi: cluster.bcg
\end{verbatim}
The output file will be cluster\_ctmdp.ctmdpi, but you can of course change
this.

\end{enumerate}



\bibliographystyle{plain}
\begin{thebibliography}{m}

\bibitem{depend-group}
Homepage of the Dependable Systems Group.\\
\url{http://depend.cs.uni-sb.de}.

\bibitem{paper-transformation}
\emph{Holger Hermanns, Sven Johr.}\\
Towards analysing nondeterministic specifications of stochastic systems

\bibitem{bcg-page}
BCG manual page.\\
\url{http://www.inrialpes.fr/vasy/cadp/man/bcg.html}

\bibitem{prism-page}
PRISM homepage.\\
\url{http://www.cs.bham.ac.uk/~dxp/prism/}

\bibitem{etmcc-page}
ETMCC homepage.\\
\url {http://www7.informatik.uni-erlangen.de/etmcc/}

\bibitem{mrmc-page}
MRMC homepage.\\
\url{http://wwwhome.cs.utwente.nl/~zapreevis/mrmc/}

\end{thebibliography}

\end{document}

